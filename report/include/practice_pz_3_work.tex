\newpage
\section{Знакомство с организацией труда программистов и пользователей ПО}

При прохождении практики на предприятии мне не запрещалось заводить разговор с программистами.
Мне не нужно было разговаривать с ними когда они заняты. Каждый раз когда я приходил в кабинет,
то на программистах были надеты наушники и они разговаривали на английском.
Кабинет с менеджером очень часто закрывался и мне не всегда удавалось что-то спрашивать.
Когда появлялся директор, то даже если кабинет с менеджером был открыт,
то там директор был в наушниках и говорил по английски.
В один момент мне сказали, чтобы я говорил с программистами на обеде в 12:00.

Нас не держали и в любой момент можно было выйти покушать, сделать чай, либо налить воды.
Так как я приходил в офис в 10:00, то я свой обед мог съедать в 14:00 и успевать поговорить в 12:00 с программистами,
чем я и занимался.

За обедом в 12:00 мне было трудно втягиваться в разговор, так как у программистов были свои темы,
которые говорили между собой. Большую часть вопросом мне уже начал отвечать один человек.
Позже я узнал, что он же является руководителем практики.

Мои вопросы по проектированию из университета никак не подтверждались,
и у меня возникали вопросы к должности проджект менеджера.
С каждым вопросом вопросов становилось ещё больше.
Эти вопросы так и остались по окончанию практики.

Технику (компьютеры) нам не выдавали. У нас в кабинете было два монитора, Play Station и проектор.
Ноутбуки мы брали с собой.
Позже я заметил, что у некоторых программистов есть личный настольный компьютер, также и у менеджера, и у директора.
На предприятии обычно использую Windows 10, Linux Ubuntu и MacOS (но именно Mac я не увидел).

Еще до прихода в офис в основной день нас разделили на группы бэкенды (10 человек) и фронтендеры (2 человека).
Позже после начала практики было ясно, что люди начали пропадать.
На фронтенд был выбран фреймворк React, а на бэкэнд фреймворк Django.

Я был в команде фронтендеров, но также и хотел быть и на серверной части.
Что и случится после окончания практики, когда я буду по новой писать код индивидуально.

Заданием было сделать регистрацию, авторизацию и создание ToDo-тасок, которые содержали заголовок описание,
тип (сделана/не сделана) и вывод их в виде календаря.

Нам был выдан ужасный макет, что и повлияло на практику и функционал, так как им неудобно было пользоваться,
когда ToDo-тасок становилось много (даже за год).
Мною уже после практики было разработано другое приложение,
которое может смотреть ToDo-таски за год, за месяц, за день и по часам.
