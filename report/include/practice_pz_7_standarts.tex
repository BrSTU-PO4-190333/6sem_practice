\newpage
\section{Ознакомление с работами по стандартизации на предприятии}

Разговаривая с программистами на кухне я также интересовался стандартизацией,
на что мне не могли ответить.
Пошла шутка про советский стандарт на розетку, что своя вилка в розетку входит,
а зарубежная вилка не подходит по размерам.
Технический программист порекомендовал спросить у директора, что я и сделал.

В 2003 году в РБ принято 16 гармонизированных государственных
стандартов, применение которых будет способствовать повышению
безопасности ИКТ. В их число входят следующие стандарты:

\begin{itemize}
  \item СТБ ИСО/МЭК 3126-2003
  «ИТ. Оценка программной продукции.
  Характеристики качества и руководство по их применению»;

  \item СТБ ИСО/МЭК 9594-1-2003
  \cite{STB_9594_1_2003_tnpa}, \cite{STB_9594_1_2003_belgiss}, \cite{STB_9594_1_2003_gostinfo}
  «ИТ. Взаимосвязь открытых систем.
  Справочник. Часть 1ю Общее описание принципов моделей услуги»;

  \item СТБ ИСО/МЭК 12119-2003
  \cite{STB_12119_2003_tnpa}, \cite{STB_12119_2003_belgiss}, \cite{STB_12119_2003_gostinfo}
  «ИТ. Пакеты программ. Требования к качеству тестирования»;

  \item СТБ ИСО/МЭК 12207-2003
  \cite{STB_12207_2003_tnpa}, \cite{STB_12207_2003_belgiss}, \cite{STB_12207_2003_gostinfo}
  «ИТ. Процессы жизненного цикла программных средств»;

  \item СТБ ИСО/МЭК 1476(!?)4-2003
  «ИТ. Сопровождение программных средств»;

  \item СТБ ИСО/МЭК 15026-2003
  \cite{STB_15026_2003_tnpa}, \cite{STB_15026_2003_belgiss}, \cite{STB_15026_2003_gostinfo}
  «ИТ. Уровни целостности систем и программных средств;
  
  \item СТБ ИСО/МЭК ТО 9294-2003
  \cite{STB_9294_2003_tnpa}, \cite{STB_9294_2003_belgiss}, \cite{STB_9424_2003_gostinfo}
  «ИТ. Руководство по управлению документированием программного обеспечения»;

  \item СТБ ИСО/МЭК ТО 12182-2003
  \cite{STB_9294_2003_tnpa}, \cite{STB_12182_2003_belgiss}, \cite{STB_12182_2003_gostinfo}
  «ИТ. Классификация программных средств».
\end{itemize}

В качестве государственных стандартов Республики Беларусь принимаются стандарты ISO, IEC, ЕЭК ООН
(Правила ЕЭК ООН), CEN (ЕN), государственные стандарты Российской Федерации (ГОСТ Р).

Если государственный стандарт идентичен международному,
то его обозначение состоит из индекса СТБ,
обозначения международного стандарта (без года его принятия),
года утверждения государственного стандарта, отделенного тире (СТБ ISO 12341-2007).
