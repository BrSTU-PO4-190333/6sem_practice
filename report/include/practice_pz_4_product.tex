\newpage
\section{Изучение конкретного программного продукта}

\textbf{Для разработки бэкэнда требуется следующее ПО}:

\begin{itemize}
  \item node js \cite{NodeJS} - для использования npm пакетов;
  \item Docker, docker-compose \cite{Docker} - для разворачивания базы данных;
  \item make \cite{make} - для того, чтобы не писать большие команды;
  \item vs Code \cite{VSCode} - редактор кода;
  \item vs Code Material Icon Theme \cite{VSCodeMaterialIconTheme} - красивые иконки в редакторе кода;
  \item vs Code Prettier \cite{VSCodePrettier} - авто выравнивание кода при сохранении;
  \item vs Code Thunder Client \cite{VSCodeThunderClient} - для отправки запросов GET, POST, PUT, DELETE, чтобы проверять работу бэкэнда при разработке;
  \item vs Code Database Client \cite{VSCodeDatabaseClient} - для подключения к базе данных и просмотра таблиц.
\end{itemize}

\hspace{0pt}

\textbf{Для разработки фронтенда требуется следующее ПО}:

\begin{itemize}
  \item firefox \cite{Firefox} - браузер;
  \item node js \cite{NodeJS} - для использования npm пакетов;
  \item make \cite{make} - для того, чтобы не писать большие команды;
  \item vs Code \cite{VSCode} - редактор кода;
  \item vs Code Material Icon Theme \cite{VSCodeMaterialIconTheme} - красивые иконки в редакторе кода;
  \item vs Code Prettier \cite{VSCodePrettier} - авто выравнивание кода при сохранении;
  \item vs Code Thunder Client \cite{VSCodeThunderClient} - для отправки запросов GET, POST, PUT, DELETE, чтобы проверять работу бэкэнда при разработке;
  \item vs Code Reactjs code snippets \cite{VSCodeReactjscodesnippets} - для того, чтобы написан пару символов развернуть шаблоный кусок кода;
  \item vs Code ESLint \cite{VSCodeESLint} - для проверки JavaScript кода и подсветки предупреждений.
\end{itemize}
