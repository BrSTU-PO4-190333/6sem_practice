\newpage
\section*{ЗАКЛЮЧЕНИЕ}
\phantomsection
\addcontentsline{toc}{section}{ЗАКЛЮЧЕНИЕ}

В ходе проведения технологической практики мною у других студентов была замечана ORM Django на Python.
Меня заинтересовали возможности этой ORM, но так как меня интересует JavaScript,
то мною была изучена ORM sequelize,
которая помогала мне писать запросы не используя SQL,
также эта ORM облегчала контролирование базы данных из-за использования миграций.
Для энпоинтов бэкэнда использовалась библиотека Express, в которой я использовал HTTP коды (200, 400, 401, 404, 500).
Также я написал документацию Swagger, чтобы можно было видеть эндпоинты, что они требуют и что дают в качестве ответа.
На основе Swagger была создана автоматическая документация Redoc.

Кроме бэкэнда я поработал с фронтендом опробовал проверку полей в форме (библиотекой react-hook-form),
отправление высплвающих сообщений (toastr) и отправку запросов на бэкэнд (async-await c fetch).

В ходе проведения технологической практики кроме фронтенда и бэкэнда мною были получены навыки dev ops'а,
так как я попробовал создавать Dockerfile'ы и писать docker-compose.
Я создал аккаунт на DockerHub и вручную загрузил образы.
На GitHub через Actions я сделал автоматическую сборку образов и Dockerfile'ов (фронтенда и бэкэнда),
которые отправляются на DockerHub автоматически, после слияния обновлений кода в ветку dev.

Мною была опробовано подключение с одного компьютера к другому используя SSH соединение Ubuntu-Ubuntu (через команду ssh),
Windows-Ubuntu (попробовав класически через программу PuTTY и через Windows Sybsystem Linux через команду ssh).
